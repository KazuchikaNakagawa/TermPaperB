\documentclass[12pt, a4paper, uplatex]{jsarticle}
\usepackage[dvipdfmx]{graphicx}
\title{アルゴリズム第2B 期末レポート}
\author{62213887 中川 和親}
\begin{document}
\maketitle

\section{概要}
本レポートでは、アルゴリズム第2Bのゲームアルゴリズムにおける
ミニマックス法とその評価関数の比較を行う。

\section{実装について}
Swiftを用いる。プログラムに際してはGitHub Copilotを活用したが、あくまで補完機能の延長線上にとどめた。
Swiftの採用理由としては、絵文字の出力が画一的に対応していることなどが挙げられる。
macOSのユーザでかつXcode Command Line Toolsをインストールしている人は標準でSwiftが使えるため、
再現性においても有利であると考えた。Linuxでも使うことができ、
今回は外部ライブラリを使わないため、環境を汚すことなく実行できる。

\section{評価関数について}
2種類用意した。
\subsection{コマの価値に準じた評価関数 Offensive関数}
私が習っていた頃のチェスにおいては、表\ref*{tab:piece_value}のような価値基準が定番であった。
\begin{table}
  \centering
  \caption{コマの価値}\label{tab:piece_value}
  \begin{tabular}{|c|c|}
    \hline
    駒     & 価値       \\
    \hline
    ポーン   & 1        \\
    ナイト   & 3        \\
    ビショップ & 3        \\
    ルーク   & 5        \\
    クイーン  & 9        \\
    キング   & $\infty$ \\
    \hline
  \end{tabular}
\end{table}
キングは無限なので、今回は十分に大きい1000とする。
自サイドのコマの価値の合計から相手サイドのコマの価値の合計を引いたものを評価値とする。
この評価関数では、相手のコマを取ることを重視することが予期されるため、
Offensive関数と名付けた。

\subsection{手数の多さの評価関数 Strategic関数}
手数が相手より多いほど評価値が高くなるようにし、キングを取ると評価値が最高、取られると評価値が最低になるようにした。
手数が多いということは、コマが中央やひらけた位置に配置できているということであり、コマを強い状態に保てていることを意味する。
また、相手の手数を少なくすることは、相手のコマを取ることにもつながるため、優秀であると考える。
コマをより多く出陣させることから、Strategic関数と名付けた。


\end{document}